% --- Template for thesis / report with tktltiki2 class ---
%
% last updated 2013/02/15 for tkltiki2 v1.02

\documentclass[11pt,finnish]{tktltiki2}

% tktltiki2 automatically loads babel, so you can simply
% give the language parameter (e.g. finnish, swedish, english, british) as
% a parameter for the class: \documentclass[finnish]{tktltiki2}.
% The information on title and abstract is generated automatically depending on
% the language, see below if you need to change any of these manually.
%
% Class options:
% - grading                 -- Print labels for grading information on the front page.
% - disablelastpagecounter  -- Disables the automatic generation of page number information
%                              in the abstract. See also \numberofpagesinformation{} command below.
%
% The class also respects the following options of article class:
%   10pt, 11pt, 12pt, final, draft, oneside, twoside,
%   openright, openany, onecolumn, twocolumn, leqno, fleqn
%
% The default font size is 11pt. The paper size used is A4, other sizes are not supported.
%
% rubber: module pdftex

% --- General packages ---

\usepackage[utf8]{inputenc}
\usepackage[T1]{fontenc}
\usepackage{lmodern}
\usepackage{microtype}
\usepackage{amsfonts,amsmath,amssymb,amsthm,booktabs,color,enumitem,graphicx}
\usepackage[pdftex,hidelinks]{hyperref}

\usepackage{setspace}
\onehalfspacing

% Automatically set the PDF metadata fields
\makeatletter
\AtBeginDocument{\hypersetup{pdftitle = {\@title}, pdfauthor = {\@author}}}
\makeatother

% --- Language-related settings ---
%
% these should be modified according to your language

% babelbib for non-english bibliography using bibtex
\usepackage[fixlanguage]{babelbib}
\selectbiblanguage{finnish}

% add bibliography to the table of contents
\usepackage[nottoc]{tocbibind}
% tocbibind renames the bibliography, use the following to change it back
\settocbibname{Lähteet}

% --- Theorem environment definitions ---

\newtheorem{lau}{Lause}
\newtheorem{lem}[lau]{Lemma}
\newtheorem{kor}[lau]{Korollaari}

\theoremstyle{definition}
\newtheorem{maar}[lau]{Määritelmä}
\newtheorem{ong}{Ongelma}
\newtheorem{alg}[lau]{Algoritmi}
\newtheorem{esim}[lau]{Esimerkki}

\theoremstyle{remark}
\newtheorem*{huom}{Huomautus}


% --- tktltiki2 options ---
%
% The following commands define the information used to generate title and
% abstract pages. The following entries should be always specified:

\title{Normalisoitu pakkausetäisyys}
\author{Timo Sand}
\date{\today}
\level{Kandidaatintutkielma}
\abstract{Referaatti}

% The following can be used to specify keywords and classification of the paper:

\keywords{avainsana 1, avainsana 2, avainsana 3}

% classification according to ACM Computing Classification System (http://www.acm.org/about/class/)
% This is probably mostly relevant for computer scientists
% uncomment the following; contents of \classification will be printed under the abstract with a title
% "ACM Computing Classification System (CCS):"
% \classification{}

% If the automatic page number counting is not working as desired in your case,
% uncomment the following to manually set the number of pages displayed in the abstract page:
%
% \numberofpagesinformation{16 sivua + 10 sivua liitteissä}
%
% If you are not a computer scientist, you will want to uncomment the following by hand and specify
% your department, faculty and subject by hand:
%
% \faculty{Matemaattis-luonnontieteellinen}
% \department{Tietojenkäsittelytieteen laitos}
% \subject{Tietojenkäsittelytiede}
%
% If you are not from the University of Helsinki, then you will most likely want to set these also:
%
% \university{Helsingin Yliopisto}
% \universitylong{HELSINGIN YLIOPISTO --- HELSINGFORS UNIVERSITET --- UNIVERSITY OF HELSINKI} % displayed on the top of the abstract page
% \city{Helsinki}
%


\begin{document}

% --- Front matter ---

\frontmatter      % roman page numbering for front matter

\maketitle        % title page
% \makeabstract     % abstract page

% \tableofcontents  % table of contents

% --- Main matter ---

\mainmatter       % clear page, start arabic page numbering

\paragraph{Kolmogorov kompleksisuus} % (fold)
\label{par:kolmogorov_kompleksisuus}
  Lyhimmän binääriohjelman pituus, joka palauttaa $x$ syötteellä $y$, on \emph{Kolmogorov kompleksisuus} $x$:stä syötteellä $y$; tämä merkitään $K(x|y)$. Pohjimmillaan Kolmogorov kompleksisuus tiedostosta on sen äärimmäisesti pakatun version pituus.

% paragraph kolmogorov_kompleksisuus (end)
\paragraph{Normalisoitu Informaatioetäisyys} % (fold)
\label{par:normalisoitu_informaatioet_isyys}

  Artikkelissa \cite{CV05} on esitelty \emph{informaatioetäisyys} $E(x,y)$, joka on määritelty lyhimpänä binääriohjelmana, joka syötteellä $x$ laskee $y$:n ja syötteellä $y$ laskee $x$:n. Tämä lasketaan seuraavasti

  \begin{align*}
    E(x,y) &= max\{K(x|y),K(y|x)\}.
  \end{align*}

  Normalisoitu versio informaatioetäisyydestä ($E(x,y)$), jota kutsutaan \emph{normalisoiduksi informaatioetäisyydeksi}, on määritelty seuraavasti

  \begin{align*}
    NID(x,y) &= \frac{ max\{K{(x|y)},K{(y|x)}\} }{ max \{K(x),K(y)\}}.
  \end{align*}

  Tätä kutsutaan \emph{samankaltaisuuden metriikaksi}, koska tämän on osoitettu \cite{CV05} täyttävän vaatimukset etäisyyden metriikaksi. \emph{NID} ei kuitenkaan ole laskettavissa tai edes semi-laskettavissa, koska Turingin määritelmän mukaan Kolmogorov kompleksisuus ei ole laskettavissa.
  Nimittäjän approksimointi annetulla kompressorilla $C$ on $max \{C(x),C(y)\}$. Osoittajan paras approksimaatio on $max\{C(xy),C(yx)\} - \min\{C(x),C(y)\}$ \cite{CV05}.
  Kun \emph{NID} approksimoidaan oikealla kompressorilla, saadaa tulos jota kutsutaan \emph{normalisoiduksi pakkausetäisyydeksi}. Tämä esitellään formaalisti myöhemmin.
% paragraph normalisoitu_informaatioet_isyys (end)

\paragraph{Normaali Kompressori} % (fold)
\label{par:normaali_kompressori}

  Seuraavaksi esitämme aksioomia, jotka määrittelevät laajan joukon kompressoreita ja samalla varmistavat \emph{normalisoidussa pakkausetäisyydessä} halutut ominaisuudet. Näihin kompressoreihin kuuluvat monet tosielämän kompressorit.

  Kompressori $C$ on \emph{normaali} jos se täyttää seuraavat aksioomat, $O(log n)$ termiin saakka:

  \begin{enumerate}
    \item \emph{Idempotenssi}: $C(xx) == C(x)$ ja $C(\lambda) = 0$, jossa $\lambda$ on tyhjä merkkijono,
    \item \emph{Monotonisuus}: $C(xy) \geq C(x)$,
    \item \emph{Symmetrisuus}: $C(xy) == C(yx)$ ja
    \item \emph{Distributiivisuus}: $C(xy) + C(z) \leq C(xz) + C(yz)$.
  \end{enumerate}
% paragraph normaali_kompressori (end)

\paragraph{Normalisoitu Pakkausetäisyys} % (fold)
\label{par:normalisoitu_pakkauset_isyys}

  Normalisoitua versiota \emph{hyväksyttävästä etäisyydestä} $E_c(x,y)$, joka on kompressoriin $C$ pohjautuva approksimaatio \emph{normalisoidusta informaatioetäisyydestä}, kutsutaan nimellä \emph{Normalisoitu Pakkausetäisyys (NCD)} \cite{CV05}. Tämä lasketaan seuraavasti

  \begin{align*}
    NCD(x,y) &= \frac{C(xy)-min\{C(x),C(y)\}}{max\{C(x),C(y)\}}.
  \end{align*}

  \emph{NCD} on funktioden joukko, joka ottaa argumenteiksi kaksi objektia (esim. tiedostoja tai Googlen hakusanoja) ja tiivistää nämä, erillisinä ja yhdistettyinä. Tämä funktioden joukko on parametrisoitu käytetyn kompressorin $C$ mukaan.

  Käytännössä \emph{NCD}:n tulos on välillä $0 \leq r \leq 1+ \epsilon$, joka vastaa kahden tiedoston eroa toisistaan; mitä pienempi luku, sitä enemmän tiedostot ovat samankaltaisia. Tosielämässä pakkausalgoritmit eivät ole yhtä tehokkaita kuin teoreettiset mallit, joten virhemarginaali $\epsilon$ on lisätty ylärajaan. Suurimmalle osalle näistä algoritmeistä on epätodennäköistä että  $\epsilon > 0.1$.
% paragraph normalisoitu_pakkauset_isyys (end)

  Luonnollinen tulkinta \emph{NCD}:stä, jos oletetaan $C(y) \geq C(x)$, on

  \begin{align*}
    NCD(x,y) = \frac{C(xy)-C(x)}{C(y)}.
  \end{align*}

  Eli etäisyys $x$:n ja $y$:n välillä on suhde $y$:n parannuksesta kun $y$ pakataan käyttäen $x$:ää, ja $y$:n pakkauksesta yksinään; suhde ilmaistaan etäisyytenä bittien lukumääränä kummankin pakatun version välillä.

  Kun kompressori on normaali niin \emph{NCD} on normalisoitu hyväksyttävä etäisyys, joka täyttää metriikan yhtälöt, eli se on samankaltaisuuden metriikka.

% --- References ---
%
% bibtex is used to generate the bibliography. The babplain style
% will generate numeric references (e.g. [1]) appropriate for theoretical
% computer science. If you need alphanumeric references (e.g [Tur90]), use
%
\bibliographystyle{babalpha-lf}
%
% instead.

% \bibliographystyle{babplain-lf}
\bibliography{references-fi}


% --- Appendices ---

% uncomment the following

% \newpage
% \appendix
%
% \section{Esimerkkiliite}

\end{document}
