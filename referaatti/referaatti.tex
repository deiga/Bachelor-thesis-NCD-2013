% --- Template for thesis / report with tktltiki2 class ---
%
% last updated 2013/02/15 for tkltiki2 v1.02

\documentclass[finnish]{tktltiki2}

% tktltiki2 automatically loads babel, so you can simply
% give the language parameter (e.g. finnish, swedish, english, british) as
% a parameter for the class: \documentclass[finnish]{tktltiki2}.
% The information on title and abstract is generated automatically depending on
% the language, see below if you need to change any of these manually.
%
% Class options:
% - grading                 -- Print labels for grading information on the front page.
% - disablelastpagecounter  -- Disables the automatic generation of page number information
%                              in the abstract. See also \numberofpagesinformation{} command below.
%
% The class also respects the following options of article class:
%   10pt, 11pt, 12pt, final, draft, oneside, twoside,
%   openright, openany, onecolumn, twocolumn, leqno, fleqn
%
% The default font size is 11pt. The paper size used is A4, other sizes are not supported.
%
% rubber: module pdftex

% --- General packages ---

\usepackage[utf8]{inputenc}
\usepackage[T1]{fontenc}
\usepackage{lmodern}
\usepackage{microtype}
\usepackage{amsfonts,amsmath,amssymb,amsthm,booktabs,color,enumitem,graphicx}
\usepackage[pdftex,hidelinks]{hyperref}

% Automatically set the PDF metadata fields
\makeatletter
\AtBeginDocument{\hypersetup{pdftitle = {\@title}, pdfauthor = {\@author}}}
\makeatother

% --- Language-related settings ---
%
% these should be modified according to your language

% babelbib for non-english bibliography using bibtex
\usepackage[fixlanguage]{babelbib}
\selectbiblanguage{finnish}

% add bibliography to the table of contents
\usepackage[nottoc]{tocbibind}
% tocbibind renames the bibliography, use the following to change it back
\settocbibname{Lähteet}

% --- Theorem environment definitions ---

\newtheorem{lau}{Lause}
\newtheorem{lem}[lau]{Lemma}
\newtheorem{kor}[lau]{Korollaari}

\theoremstyle{definition}
\newtheorem{maar}[lau]{Määritelmä}
\newtheorem{ong}{Ongelma}
\newtheorem{alg}[lau]{Algoritmi}
\newtheorem{esim}[lau]{Esimerkki}

\theoremstyle{remark}
\newtheorem*{huom}{Huomautus}


% --- tktltiki2 options ---
%
% The following commands define the information used to generate title and
% abstract pages. The following entries should be always specified:

\title{Referaatti: Clustering by Compression}
\author{Timo Sand}
\date{\today}
\level{Kandidaatintutkielma}
\abstract{Referaatti.}

% The following can be used to specify keywords and classification of the paper:

\keywords{avainsana 1, avainsana 2, avainsana 3}

% classification according to ACM Computing Classification System (http://www.acm.org/about/class/)
% This is probably mostly relevant for computer scientists
% uncomment the following; contents of \classification will be printed under the abstract with a title
% "ACM Computing Classification System (CCS):"
% \classification{}

% If the automatic page number counting is not working as desired in your case,
% uncomment the following to manually set the number of pages displayed in the abstract page:
%
% \numberofpagesinformation{16 sivua + 10 sivua liitteissä}
%
% If you are not a computer scientist, you will want to uncomment the following by hand and specify
% your department, faculty and subject by hand:
%
% \faculty{Matemaattis-luonnontieteellinen}
% \department{Tietojenkäsittelytieteen laitos}
% \subject{Tietojenkäsittelytiede}
%
% If you are not from the University of Helsinki, then you will most likely want to set these also:
%
% \university{Helsingin Yliopisto}
% \universitylong{HELSINGIN YLIOPISTO --- HELSINGFORS UNIVERSITET --- UNIVERSITY OF HELSINKI} % displayed on the top of the abstract page
% \city{Helsinki}
%


\begin{document}

% --- Front matter ---

\frontmatter      % roman page numbering for front matter

\maketitle        % title page
% \makeabstract     % abstract page

% \tableofcontents  % table of contents

% --- Main matter ---

\mainmatter       % clear page, start arabic page numbering

\section{Normalisoitu Pakkausetäisyys} % (fold)
\label{sec:normalisoitu_pakkauset_isyys}

  \emph{Normalisoitu Pakkausetäisyys (NCD)} on funktioden perhe jotka ottavat argumenteiksi kaksi objektia (esim. tiedostoja, Googlen haku sanoja) ja evaluoivat määrätyn kaavan, joka ilmaisee tiivistetyn version näistä objekteista, erillisinä ja yhdistettynä. Täten tämä funktioden perhe on parametrisoitu käytettävän kompressorin mukaan.

  \paragraph{Normalisoitu Pakkausetäisyys} % (fold)
  \label{par:normalisoitu_pakkauset_isyys}

    Normalisoitua versio hyväksyttävästä etäisyydestä $E_c(x,y)$, joka on kompressoriin $C$ pohjautuva approksimaatio \emph{Normalisoidusta Informaatioetäisyydestä (NID)}, kutsutaan nimellä \emph{Normalisoitu Pakkausetäisyys (NCD)} \cite{CV05}
  \\\\
    Jos $x$ ja $y$ ovat kaksi objektia, ja $C(x)$ on $x$ kompressoitu etäisyys  käyttäen kompressoria $C$, sitten NCD määräytyy seuraavanlaisesti

    \begin{align*}
      NCD(x,y) &= \frac{C(xy)-min\{C(x),C(y)\}}{max\{C(x),C(y)\}}
    \end{align*}

    Oletamme että objektit ovat äärellisiä merkkijonoja 0:sta ja 1:stä. Jokainen tiedosto tietokoneella on tätä muotoa. Voidaan määritellä informaatioetäisyys merkkijonojen $x$ ja $y$ välillä lyhimmän ohjelman $p$ mukaan, joka laskee $x$:n $y$:stä ja toisinpäin. Tämä lyhin ohjelma on määrätyssä ohjelmointikielessä. Teknisistä syistä käytetään Turing koneitten teoreettista käsitettä.
    $p$:n pituuden ilmaisemiseen käytetään \emph{Kolmogorov kompleksisuuden} käsitettä. On osoitettu, että
    $$|p| = max\{K(x|y),K(y|x)\}$$

    Käytännössä \emph{NCD}:n tulos on $ \leq r \leq 1+ \epsilon$ joka kuvastaa kahden tiedoston erotusta. Pienempi luku tarkoittaa enemmän samankaltaisuutta. $\epsilon$ on virhemarginaali korjaamaan tosielämän pakkausalgoritmejen puutteita, suurimmalle osalle pakkausalgritmeistä on epätodennäköistä nähdä $\epsilon \geq 0.1$
  % paragraph normalisoitu_pakkauset_isyys (end)

  \paragraph{Normalisoitu Informaatioetäisyys} % (fold)
  \label{par:normalisoitu_informaatioet_isyys}
    $$ NID(x,y) = \frac{ \max\{K{(x|y)},K{(y|x)}\} }{ \max \{K(x),K(y)\}} $$,

    jossa $K(x|y)$ on algoritmin tietoa $x$:stä kun $y$ on syöte. \emph{NID}:iä kutsutaan \emph{samankaltaisuuden metriikaksi}, koska $NID(x,y)$ on osoitettu täyttävän vaatimukset etäisyyden metriikaksi. \emph{NID} ei kuitenkaan ole laskettavissa tai edes semi-laskettavissa. \emph{NCD} on siis käytännön versio \emph{NID}:stä
  % paragraph normalisoitu_informaatioet_isyys (end)
  % section normalisoitu_pakkauset_isyys (end)

  \emph{NCD}:stä on luonnollinen tulkinta. Jos oletetaan $C(y) \geq C(x)$, niin voimme kirjoittaa
  $$ NCD(x,y) = \frac{C(xy)-C(x)}{C(y)} $$

  Eli siis etäisyys $NCD(x,y)$ $x$:n ja $y$:n välillä tuottaa parannuksen kun pakataan $y$ käyttäen $x$:ää kuten aikaisemmin pakattuna ``tietokantana'' ja pakkaamalla $y$ tyhjästä, ilmaistuna suhteena pituuden biteissä välillä kummastakin pakatusta versiosta.

% --- References ---
%
% bibtex is used to generate the bibliography. The babplain style
% will generate numeric references (e.g. [1]) appropriate for theoretical
% computer science. If you need alphanumeric references (e.g [Tur90]), use
%
\bibliographystyle{babalpha-lf}
%
% instead.

% \bibliographystyle{babplain-lf}
\bibliography{references-fi}


% --- Appendices ---

% uncomment the following

% \newpage
% \appendix
%
% \section{Esimerkkiliite}

\end{document}
